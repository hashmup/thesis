\section{結論}
本研究では神経回路シミュレーションソフトであるNEURONを対象として,
シミュレーションの自動最適化を行うプログラムを開発することで計算性能の向上を図った.
以下, 本研究による成果と課題点, そして本研究を活用した将来的な展望について述べる.\\

\subsection{成果}
本研究では, シミュレーションを実行するマシンの環境的な差異を減らすために仮想化環境を導入するための環境設定スクリプト,
パラメータを推定するために複数回のシミュレーションを行いその結果を集約するためのシミュレータ
そしてシミュレータから渡されたパラメータを用いてMODファイルからC言語ファイルを生成するトランスパイラの開発を行った.\\
 また, シミュレーションのパラメータ定義の追加や削除が容易となるようJSON形式を用いて定義ファイルを作成した.
さらにC言語の生成に対して最適化を行う上で, 新規の最適化アルゴリズムを実装・追加する場合にコードの変更量が少なくてすむようなファイル構成で実装を行った.\\
 小規模のシミュレーションを通して行った最適化のためのパラメータ推定では, MPIプロセス数, OpenMPスレッド数, SIMD化, 配列のくくり出しの4つのパラメータ
に対し最適化を行い, 先行研究を通して手動での最適化を行ったNEURONを利用した場合にはわずかに及ばないものの,
最適化を行っていないNEURONで実行した場合と比較して15〜30倍近い性能向上が見られるなど,
本研究のプログラムを用いることで十分実用的に神経回路シミュレーションの実行を高速化できることを示した.\\

\subsection{課題点}
課題点としては, 第一に小規模なシミュレーションと大規模なシミュレーションではリソースの利用状況が大きく異なるため,
実行マシンに関わるパラメータに関しての推定結果が大規模なシミュレーションに対して最適化されているとは言えないということがあげられる.
これはクラスタのようにノード数が少ないマシンに対する影響はないが, 京のように大量のノードを利用できる場合には大きなボトルネックとなりうる.
そのため細胞数とコンパートメント数を非常に多くした上でどこまでシミュレーション時間を減らせるのかを見極め,
規模は大きいが実行時間は短いベンチマークを作成することが必要となると考えられる.\\
 第二に, パラメータの推定がその候補に対し全探索で行われているためシミュレーション規模を大きくするとパラメータ推定にかかる時間が非常に長くなってしまうことも
解決すべき課題である. 特に本論文執筆時の実装ではそれぞれのパラメータの候補数を掛け合わせたものが全候補の数となるため,
パラメータの種類に対して指数的に候補の数が増えてしまうことから最適になりえない候補を適切に除外するアルゴリズムが必要となる.\\
 最後の課題点として, 本論文執筆時の実装ではMODファイルからC言語を生成するトランスパイラは限定的な対象にしか用いるコトができないことがあげられる.
特に単体のMODファイルで完結せず複数のMODファイルが関与する場合適切なトランスパイルを行うことができないため,
トランスパイラを汎用化させることは必須となる.\\

\subsection{将来的な展望}
最後に将来的な展望について述べる.
まず課題点であげた, 大規模シミュレーションに対応するためのベンチマーク, シミュレーションアルゴリズムの改善そしてトランスパイラの汎用化を通して
シミュレータの適用範囲できる対象を拡大することで, 手動での最適化を必要とせずシミュレーションの十分な高速化が行えると考えられる.\\
 また, 本研究においてはシミュレーションの高速化のみに着目してシミュレータの開発を行っていたが,
最適化に用いるパラメータだけでなくシミュレーションそのものの内容に関与するパラメータの定義も行えるようにすることで,
高速化を行った上でシミュレーション自体のパラメータ推定も並行して行えるようになることが期待される.\\
