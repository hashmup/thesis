\section*{謝辞}
\addcontentsline{toc}{section}{謝辞}
本研究は, 情報理工学系研究科知能機械情報学専攻の神崎亮平教授のご指導のもと行われました.\\
神崎亮平教授には, 研究だけでなく大学院進学や就職といった自分の進路に関して言葉をかけてくださり, 精神的な面で支えていただきました.\\
 高橋宏知講師には, 研究室見学の際にそれまで全く知見のなかった神経科学について丁寧に説明していただき, この研究室に所属したいと思うきっかけをいただきました.\\
 微小脳グループのリーダーである加沢知毅氏には, 研究内容だけでなく発表の場や卒論執筆を通して非常に多くの助言をいただきました.
研究で行き詰まっている際にいただいた助言が解決のきっかけになったことは数え切れません. 論文の校正や発表練習にも熱心に行っていただき実に様々な面で助けていただきありがとうございました.\\
 本研究は博士課程の宮本大輔さんの修士論文から発展したものであったため, 宮本さんからは研究の始め方や参考になる資料など研究を進める上で必要な数々の情報を教えていただきました.
ご自身も博士論文でお忙しい中, 私が研究に詰まった際はその都度的確な回答をいただきました.
また, 研究だけでなく外部大学との勉強会に誘ってもらったり, チームに現在いらっしゃる方々や過去に卒業された方々との交流の場を設けてくださったため非常に楽しく過ごすことができました.\\
 東京大学先端科学技術センターの Haupt Stephan Shuichi 氏には, 神経回路についての知見をいただいた他, 海外の院への進学を考えていた際には快く英語の練習にも付き合っていただきました.\\
 修士課程の角田さんには, 中間発表の際にアブストの添削をしていただいたり院試のアドバイスをいただいたりしただけでなく, 研究以外のくだらない話にも付き合っていただきとても楽しかったです.\\
 電気通信大学の山崎匡准教授には, ARM Assemblyに関するゼミに参加させていただき, 京などの大規模並列計算機での最適化を行う上での知見を多くいただきました.\\
 研究室の秘書をされている木村氏, 岩月氏には, 通常の事務手続きだけでなく奨学金の申請についても大変お世話になりました.\\
 最後に, この1年間の卒業研究を支えてくださった神崎・高橋研究室の皆様と家族, 友人を含めすべての方にこの場を借りて厚く御礼を申し上げます.

\begin{flushright}
 平成30年2月2 井上 裕太
\end{flushright}
\clearpage
