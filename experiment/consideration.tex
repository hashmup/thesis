\section{考察}
\ref{sec:sim-result}章に示したシミュレーション結果について,
その結果から得られた各パラメータとシミュレーションの実行時間について考察する.\\
また, 最適化なしのNEURONと先行研究にて手動での最適化を行ったNEURON, (そしてICCを用いてコンパイルしたNEURON)を用いてシミュレーションを行い,
本研究での自動最適化後もっとも高速化されていたシミュレーション結果と比較することで自動最適化の効果を評価・考察する.\\

\subsection{MPIプロセス数とOpenMPスレッド数}
1000msまでのシミュレーションの最適化において, MPIプロセス数は非常に大きな役割を果たすパラメータであったと言える.\\
クラスタと京双方の環境においてMPIのプロセス数をコア数に近くすることでシミュレーションの高速化がなされたが,
これはハイブリッド並列の仕組み上先にMPIプロセスが生成されたのちにOpenMPのスレッドが生成されるため,
MPIプロセスによってコアを占有することでスレッド生成のコストが抑えられたためであると考えられる.\\
またこの結果からハイブリッド並列に関与するパラメータは小規模なシミュレーションから求めることはできないということもわかった.
これはMPIプロセスが必要とする通信リソースを小規模なシミュレーションでは使い切れないためであり,
この問題を解決するためには細胞数を現在の256から大幅に増やした状態でのシミュレーションを行う必要があるが,
その場合一度のシミュレーションにかかる時間も比例して大きくなるためパラメータ推定にかかる時間が膨大になるという新たな問題が生じる.\\
そのためシミュレータの生成するコードやパラメータ推定の方法をより効率化する必要があるが, これは今後の課題としたい.\\

\subsection{SIMD化と配列のくくり出し}
コンパイラによるSIMD化はMPIプロセス数と並び高速化に貢献したパラメータであり,
\ref{subsubsec:simd}で示した理論値性能とまではいかないもの, 実行時間は半分以上縮まることがわかった.\\
特に京ではデフォルトのコードとSIMD化を行ったコードとの間で差が大きく出ており, これはSIMD化に強い京コンパイラの貢献が大きいと考えられる.\\
一方で, 配列のくくり出しに関してはシミュレーションの規模が大きくなるにつれSIMD化のみを行ったものよりも計算性能が落ちているが,
これは配列のくくり出しそのものに意味がないというより,
探索する対象のパラメータの組を減らすため配列のくくり出しを行うか否かの2通りの試行しか行わなかったことが大きな原因であると考えられる.\\
ゆえに, より多くのパラメータ候補を試行することがそれぞれの最適化アルゴリズムを有効利用するためには必要であり,
ハイブリッド並列化と同様にパラメータ推定の過程そのものを効率化することが詳細な最適化を行う上で必要となると考えられる.\\

\subsection{自動最適化の評価}
自動最適化の評価は表\ref{table:sim-cond}に示した条件で行った.\\
\begin{table}[htb]
{\footnotesize
  \caption {シミュレーション条件}
  \begin{center}
    \begin{tabular}{|c|l|}
      \hline
      パラメータ & 値の範囲\\ \hline
      シミュレーション時間 & 1000\\ \hline
      神経細胞数 & 256\\ \hline
      ネットワーク & リングネットワーク, ランダムネトワーク\\
                 & Watts and Strogatzネットワーク\\ \hline
      試行回数 & 10回 \\ \hline
    \end{tabular}
    \label{table:sim-cond}
  \end{center}
}
\end{table}

また自動最適化の結果として用いたパラメータは,
\ref{subsec:detail-sim}においてシミュレーション時間1000msで行ったシミュレーションの内,
平均の実行時間がもっとも短いパラメータの組を用いる.\\
( TODO: openmpの値を更新する)
\begin{table}[htb]
  \caption {自動最適化パラメータ}
  \begin{center}
    \begin{tabular}{|c|l|l|}
      \hline
      パラメータ & クラスタでの値 & 京での値\\ \hline
      MPIプロセス数 & 26 & 8\\ \hline
      OpenMPスレッド数 & 10 & 10\\ \hline
      SIMD化 & 行う & 行う\\ \hline
      配列のくくり出し & 行わない & 行わない \\ \hline
    \end{tabular}
    \label{table:auto-tuned-param}
  \end{center}
\end{table}
