\section{考察}
\ref{sec:sim-result}章に示したシミュレーション結果について,
その結果から得られた各パラメータとシミュレーションの実行時間について考察する.\\

\subsection{MPIプロセス数とOpenMPスレッド数}
1000msまでのシミュレーションの最適化において, MPIプロセス数は非常に大きな役割を果たすパラメータであったと言える.\\
 クラスタと京双方の環境においてMPIのプロセス数をコア数に近くすることでシミュレーションの高速化がなされたが,
これはハイブリッド並列の仕組み上先にMPIプロセスが生成されたのちにOpenMPのスレッドが生成されるため,
MPIプロセスによってコアを占有することでスレッド生成のコストが抑えられたためであると考えられる.\\
 またこの結果からハイブリッド並列に関与するパラメータは小規模なシミュレーションから求めることはできないということもわかった.
これはMPIプロセスが必要とする通信リソースを小規模なシミュレーションでは使い切れないためであり,
この問題を解決するためには細胞数を現在の256から大幅に増やした状態でのシミュレーションを行う必要があるが,
その場合一度のシミュレーションにかかる時間も比例して大きくなるためパラメータ推定にかかる時間が膨大になるという新たな問題が生じる.\\
 そのためシミュレータの生成するコードやパラメータ推定の方法をより効率化する必要があるが, これは今後の課題としたい.\\

\subsection{SIMD化と配列のくくり出し}
コンパイラによるSIMD化はMPIプロセス数と並び高速化に貢献したパラメータであり,
\ref{subsubsec:simd}項で示した理論値性能とまではいかないもの, 実行時間は半分以上縮まることがわかった.\\
 特に京ではデフォルトのコードとSIMD化を行ったコードとの間で差が大きく出ており, これはSIMD化に強い京コンパイラの貢献が大きいと考えられる.\\
 一方で, 配列のくくり出しに関してはシミュレーションの規模が大きくなるにつれSIMD化のみを行ったものよりも計算性能が落ちているが,
これは配列のくくり出しそのものに意味がないというより,
探索する対象のパラメータの組を減らすため配列のくくり出しを行うか否かの2通りの試行しか行わなかったことが大きな原因であると考えられる.\\
 ゆえに, より多くのパラメータ候補を試行することがそれぞれの最適化アルゴリズムを有効利用するためには必要であり,
ハイブリッド並列化と同様にパラメータ推定の過程そのものを効率化することが詳細な最適化を行う上で必要となると考えられる.\\

\subsection{シミュレーションの規模とパラメータの関係}
一方で, シミュレーション時間100msで行った小規模なシミュレーションにおける各パラメータとシミュレーション時間1000msで行った場合を比較すると,
本研究で用いたパラメータについてはどのパラメータにおいても一定の相関が見られることがわかる.
このことから, 規模を大きくすることで特定のリソースを使い切ってしまうパラメータでないならば小規模のシミュレーションから最適なパラメータ推定を十分に行えると考えられる.\\
 さらに, リソースに関連するパラメータは実行マシンに関わるパラメータであるため,
この事実を利用することでパラメータの候補を実行マシンに関与するパラメータとモデルに関与するパラメータに2分し別々に最適化を行うことも可能であり,
組み合わせによって膨大となっているパラメータ候補数を減らすことが可能であると考えられる.\\

\subsection{本研究の評価}
実行時間という観点では, \ref{subsec:compare}節において示したように本研究で開発したプログラムを利用することで,
手動最適化を行った場合と近い高速化を行えると考えられる.\\
 また, 最適化に用いたパラメータの一つである配列のくくり出しについては探索を行うパラメータの候補が多くなりすぎるという理由で\ref{subsec:small-sim}においては配列のくくり出しを行うか行わないかの2通りのみを対象としていたが,
\ref{subsubsec:soa}で述べたように配列のくくり出しは本来コンパイラによるSIMD化と適切に組み合わせることで効果を持つものである.\\
 そこで附録として, \ref{subsec:compare}節で用いたパラメータのみを対象として, 配列のくくり出しを確かめるべく追加でシミュレーションを行った.\\
