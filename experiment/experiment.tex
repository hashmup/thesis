\section{その他もろもろ}
\subsection{箇条書き}
以下のように箇条書き用にいくつかの環境が用意されている.
\begin{itemize}
 \item 運動の第一法則とは,慣性の法則のことです.
 \item 運動の第二法則とは,運動方程式のことです.
 \item 運動の第三法則とは,作用反作用の法則のことです.
\end{itemize}

\begin{itemize}
 \item[$\clubsuit$] 運動の第一法則とは,慣性の法則のことです.
 \item[$F=ma$] 運動の第二法則とは,運動方程式のことです.
 \item[文字もOK] 運動の第三法則とは,作用反作用の法則のことです.
\end{itemize}

\begin{enumerate}
 \item 運動の第一法則とは,慣性の法則のことです.
 \item 運動の第二法則とは,運動方程式のことです.
 \item 運動の第三法則とは,作用反作用の法則のことです.
\end{enumerate}

\begin{description}
 \item[運動の第一法則]慣性の法則のことです.
 \item[運動の第二法則]運動方程式のことです.
 \item[運動の第三法則]作用反作用の法則のことです.
\end{description}

入れ子も可能.
\begin{itemize}
 \item 運動の第一法則とは,慣性の法則のことです.
    \begin{itemize}
      \item 慣性系とは,運動方程式が成り立つ座標系のことです.
      \item ガリレイ系と呼ぶこともあります.
      \item 多くの場合,地球に固定された座標系(実験室系)は良い近似です.
     \end{itemize}
 \item 運動の第二法則とは,運動方程式のことです.
 \item 運動の第三法則とは,作用反作用の法則のことです.
\end{itemize}

\subsection{マクロ}
自分で新しい命令を作れる機能.
プリアンブル中の\verb+\newcommand+で定義している.

\begin{itemize}
 \item \verb+\newcommand{\hoge}{\textcolor{red}{ほげほげ}}+
 \item \verb+\newcommand{\Hoge}[2]{\textcolor{#1}{#2#2}}+
 \item \verb+\newcommand{\HOGE}[2][red]{\textcolor{#1}{#2#2}}+
\end{itemize}
で\verb+\hoge,\Hoge,\HOGE+の3つの命令を定義した.
\begin{itemize}
 \item \verb+\hoge+ → \hoge
 \item \verb+\Hoge{green}{ふが}+ → \Hoge{green}{ふが}
 \item \verb+\HOGE{ふが}+ → \HOGE{ふが}
 \item \verb+\HOGE[blue]{ふが}+ → \HOGE[blue]{ふが}
\end{itemize}
のようになる.
\verb+[]+はオプション引数なので省略できる.

既存の命令を上書きできる\verb+renewcommand+というのもあるが今回は割愛する.
\end{document}
