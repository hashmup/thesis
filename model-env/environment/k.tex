\paragraph{京でのジョブの実行}~\\
\begin{table}[htb]
  \caption {京でのジョブ関連コマンド}
  \begin{center}
    \begin{tabular}{|c|p{12cm}|}
      \hline
      コマンド & 説明 \\ \hline
      pjsub & pjsub \"サブミットするスクリプトのパス\"とすることでジョブをキューシステムに登録し,ジョブIDを出力する.\\ \hline
      pjdel & pjdel \"ジョブID\"とすることで現在実行中または待機中のジョブを停止・削除する.\\ \hline
      pjstat & 現在実行または待機中のジョブの一覧を表示する\\ \hline
    \end{tabular}
  \end{center}
\end{table}

{\footnotesize
\lstinputlisting[caption=京のジョブスクリプト例, label=k-job-script,frame=single]{src/job/k-job}
}

{\footnotesize
\begin{lstlisting}[caption=京でのコマンド実行例,label=k-job-example,numbers=none]
$ pjsub job.sh
[INFO] PJM 0000 pjsub Job 7129316 submitted.

$ pjstat
ACCEPT QUEUED  STGIN  READY RUNNING RUNOUT STGOUT   HOLD  ERROR   TOTAL
     0      1      0      0       0      0      0      0      0       1
s      0      1      0      0       0      0      0      0      0       1

JOB_ID   JOB_NAME  MD  ST   USER      GROUP      START_DATE       ELAPSE_TIM  NODE_REQUIRE    RSC_GRP  SHORT_RES
7129316  job.sh    NM  QUE  user    group  [--/-- --:--:--]  0000:00:00      1:-         small    -
\end{lstlisting}
}

\begin{table}[htb]
  \caption {京でのジョブの状態}
  \begin{center}
    \begin{tabular}{|c|p{12cm}|}
      \hline
      ジョブのステータス & 説明 \\ \hline
      QUE & ジョブキューで待機中.\\ \hline
      STI & ジョブの実行に必要なファイルをステージインしている.\\ \hline
      RUN & ジョブを実行中.\\ \hline
      STO & ジョブの実行結果をステージアウトしている.\\ \hline
    \end{tabular}
  \end{center}
\end{table}
