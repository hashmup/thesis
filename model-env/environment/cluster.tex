\paragraph{クラスタでのジョブの実行}~\\
\begin{table}[htb]
  \caption {クラスタでのジョブ関連コマンド}
  \begin{center}
    \begin{tabular}{|c|p{12cm}|}
      \hline
      コマンド & 説明 \\ \hline
      qsub & qsub \"サブミットするスクリプトのパス\"とすることでジョブをキューシステムに登録し,ジョブIDを出力する.\\ \hline
      qdel & pjdel \"ジョブID\"とすることで現在実行中または待機中のジョブを停止・削除する.\\ \hline
      qstat & 現在実行または待機中のジョブの一覧を表示する\\ \hline
    \end{tabular}
  \end{center}
\end{table}

{\footnotesize
\lstinputlisting[caption=クラスタのジョブスクリプト例, label=cluster-job-script,frame=single]{src/job/cluster-job}
}

{\footnotesize
\begin{lstlisting}[caption=クラスタでのコマンド実行例,label=cluster-job-example,numbers=none]
$ qsub job.sh
20252.cluster.localdomain

$ qstat
>> qstat
Every 1.0s: qstat                                                                                                                                            Wed Jan 10 01:06:06 2018

Job ID                    Name             User            Time Use S Queue
------------------------- ---------------- --------------- -------- - -----
20251.cluster              job.sh        inoue           00:05:38 C cluster
20252.cluster              job.sh        inoue                  0 R cluster
\end{lstlisting}
}

\begin{table}[htb]
  \caption {クラスタでのジョブの状態}
  \begin{center}
    \begin{tabular}{|c|p{12cm}|}
      \hline
      ジョブのステータス & 説明 \\ \hline
      Q & ジョブキューで待機中.\\ \hline
      R & ジョブを実行してる.\\ \hline
      C & ジョブが完了した.\\ \hline
    \end{tabular}
  \end{center}
\end{table}
