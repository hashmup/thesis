\paragraph{クラスタでのジョブの実行}~\\
\begin{table}[htb]
  \begin{center}
    \caption {クラスタでのジョブ関連コマンド}
    \begin{tabular}{|c|p{12cm}|}
      \hline
      コマンド & 説明 \\ \hline
      qsub & qsub サブミットするスクリプトのパス\\
           & とすることでジョブをキューシステムに登録し,ジョブIDを出力する.\\ \hline
      qdel & pjdel ジョブID\\
           & とすることで現在実行中または待機中のジョブを停止・削除する.\\ \hline
      qstat & 現在実行または待機中のジョブの一覧を表示する\\ \hline
    \end{tabular}
  \end{center}
\end{table}~\\

{\footnotesize
\lstinputlisting[title=クラスタのジョブスクリプト例, label=cluster-job-script,frame=single]{src/job/cluster-job}
}
\vspace{1cm}
\begin{table}[htb]
  \begin{center}
  \title {クラスタでのコマンド実行例}
{\footnotesize
\begin{framed}
\begin{verbatim}
$ qsub job.sh
20252.cluster.localdomain

$ qstat
>> qstat
Every 1.0s: qstat                                                                                                                                            Wed Jan 10 01:06:06 2018

Job ID                    Name             User            Time Use S Queue
------------------------- ---------------- --------------- -------- - -----
20251.cluster              job.sh        inoue           00:05:38 C cluster
20252.cluster              job.sh        inoue                  0 R cluster
\end{verbatim}
\end{framed}
}
\end{center}
\end{table}
% \caption{クラスタでのコマンド実行例}
\clearpage
\begin{table}[htb]
  \begin{center}
    \caption {クラスタでのジョブの状態}
    \begin{tabular}{|c|p{12cm}|}
      \hline
      ジョブのステータス & 説明 \\ \hline
      Q & ジョブキューで待機中.\\ \hline
      R & ジョブを実行してる.\\ \hline
      C & ジョブが完了した.\\ \hline
    \end{tabular}
  \end{center}
\end{table}
