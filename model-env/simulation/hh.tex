\subsubsection{Hodgkin-Huxleyマルチコンパートメントモデル}
\paragraph{Hodgkin-Huxleyモデル}~\\
神経系における情報の伝達は, ニューロンの電気的な活動によって行われる.
こうしたニューロンの電気的活動を表現する方法として,
1952年にHodgkinとHuxleyによってヤリイカの神経の活動電位の研究を基にした微分方程式のモデルが考案された.\\
Hodgkin-Huxleyモデルでは, 各イオンに対するニューロンの細胞膜の等価性を基に, ニューロンの電気的活動を微分方程式を用いて表現している.\\
他のモデルと比べ計算量が多い一方, 実際の生物の神経系の働きに近くたいていのイオンチャネルのモデルを表すことができるという特徴を持っている.\\

\subparagraph{モデルの定式化}
ニューロンはイオンを通さない脂質二重膜によって構成され, 特定のイオンを選択的に透過させるチャネルと呼ばれるタンパク質が膜上に分布している.

\paragraph{マルチコンパートメントモデル}
