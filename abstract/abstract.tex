\section{序論}
\subsection{研究の背景}
脳機能の理解を目的として,スーパコンピュータを用いた神経回路のシミュレーションが行われている. また,消費電力やシミュレーションの割り当て時間といったリソースの問題やリアルタイムデータ同化への需要から
シミュレーションの高速化・最適化が求められている. しかし,神経細胞には様々な種類のものが存在するため,個々の神経細胞のイオンチャンネルのモデルを最適化された形で実装するために,これまでそれぞれのモデルに対して多大な努力が行われてきた.
また,現代の計算機にも多様な種類が存在し,それぞれに対する最適化も個別に行われてきた.
本研究の目的はそれぞれの細胞モデルのシミュレーションコードを個々のアーキテクチャに合わせて,自動又は半自動的に最適化を行う手法を確立することである.

\subsection{研究の目的と手法}
通常はソース内で
何回改行しようと
このように
出力
結果で改行は
起こらない.
改行するには\\
\verb+\+\verb+\+や\verb+\newline+を用いる.

また,このようにソースで一行空けると
改段落が発生する.自動的に字下げされているよね.\verb+\par+でも同じ.\par
字下げを明示的に指定するには\verb+\indent+や\verb+\noindent+を使う.

\noindent このようにインデントが抑制される.

\verb+\newpage+をというコマンドもあり,使うと{\newpage}こうなる.

\subsection{本論文の構成}
本論文は全5章から構成される.
本章では本研究の背景・目的について述べた.
第2章では,実験方法について述べる.
