\label{sec:algorithm}
本研究では,モデルに依存するパラメータと実行マシンに依存するパラメータ,そしてプログラムのコンパイル時に関わるパラメータ(コンパイルオプション)を調節することでシミュレーション系の最適化を目指した.\\
以下にそれぞれのパラメータの詳細を示す.\\
\subsection{モデルに依存するパラメータ}
以下にHodgkin-Huxley方程式のモデルを例としてそれぞれのパラメータを示す.\\
モデルに依存するパラメータに関しては先行研究\cite{miyamoto-master}においてSIMD化, 配列構造の最適化により計算速度が大きく向上することが示されているため,
その二つに加え配列構造の順序を入れ替えることによってキャッシュヒット率の向上に取り組んだ.\\
Hodgkin-Huxley方程式は,NEURON内においてMOD形式で次のように記述されている.\\
{\footnotesize
\lstinputlisting[caption=hh.mod,label=mod-hh,frame=single]{src/mod/hh.mod}
}
先行研究の中でも示されている通り,この計算式の中で多くの計算時間を必要とするのはDERIVATIVEであり,
以下のパラメータの多くはDERIVATIVEの計算を行う上でキャッシュヒット率をあげることを目的としている.\\
\subsubsection{SIMD化}
・宮本さんの論文を参照
また,SIMD化を行う方法としてavxやinline asmなどを利用する方法も存在する.\\
これは現在の課題である汎用性を達成した上で,汎用性を崩さないよう取り組む内容であるため,
今後の課題としたい.\\


\subsubsection{配列構造}
配列を複数定義し,一つの計算の中で呼び出す場合空間的局所性が低くなり,キャッシュミスを多く生じさせる可能性がある.
そのため,同時に利用する配列達を一つの配列としてくくりだすことで空間的局所性を高くし,高速化を図る手法が考えられる.\\
{\footnotesize
\begin{lstlisting}[numbers=none]
int a[100], b[100], c[100], d[100];
for i=0 to 100 {
  d[i] = a[i] + b[i] + c[i];
}
\end{lstlisting}
}
というプログラムを例とすると,
{\footnotesize
\begin{lstlisting}[numbers=none]
int abcd[100][4];
for i=0 to 100 {
  abcd[i][3] = abcd[i][0] + abcd[i][1] + abcd[i][2];
}
\end{lstlisting}
}
とすることで連続した領域にアクセスさせることができるようになり,キャッシュ効率の向上が見込まれる.\\
一方で,SIMD化の観点では配列の各要素がバラバラになってしまい非連続的になるためSIMD命令を使いにくくなる可能性もある.\\

\begin{figure}[htb]
% h:here, t:top, b:bottom, p:page
  \begin{center}
    \includegraphics[width=10.0cm, angle=-90]{./images/simd_soa.pdf}
    \caption{SIMD化と配列のくくりだし}
    \label{fig:simd-soa}
  \end{center}
\end{figure}~\\
以上から空間局所性とSIMD化という最適化を行う上で大変重要な要素どちらか一方だけを考えるのではなく,
適切なハイブリッド構造を用いることを目指した.\\
( TODO: 事実確認)
演算に利用する変数の数が多い時はSIMD化は難しい. これはSIMD演算器のビット数に依存する問題だが倍精度の演算をする場合
1-4変数(double型64bitsに対しSIMD演算器は64-256bitsが一般的)の演算を並列処理できるが,変数の数がより多い場合は
同時に実行することが不可能であるからである.\\
この場合,配列をくくり出すことによって空間局所性を高くする方がより計算処理の高速化を実現できると考えられる.\\
また,配列をくくり出す際相互に関係のない配列を一つにまとめてもキャッシュラインに入りきらなくなるなどの問題が発生する.\\

以上を踏まえ,SIMD化と配列のくくりだしのハイブリッドを実現するために以下のアルゴリズムを実装した.\\
{\footnotesize
\lstinputlisting[caption=SIMD化と配列のくくりだしのアルゴリズム 疑似コード,label=pseudocode-simd-soa,frame=single]{src/pseudocode/simd-soa}
}~\\

\paragraph{Union-Find木の利用}~\\
MODファイル内で
\begin{eqnarray}
  a &=& b * c\\
  d &=& e * f\\
  g &=& a + h
\label{eq:3x0}
\end{eqnarray}
と記述されていたとすると,各ステップの計算において関連しうる変数は(a, b, c, g, h)と(d, e, f)であることがわかる.\\
そのため,相互に関連しえない変数に関しては互いに影響を及ぼさないため,
グループとしてまとめ切り離して考えることができる.\\
このグループを作成するにあたり,Union-Find木を用いてMODファイルに定義された式の変数を分類した.\\
こうしてUnion-Find木で作成されたグループは,空間局所性またはSIMD化のどちらかがより有効に働くことから,
それぞれのグループを配列としてくくり出すか否かを独立に試行することで空間局所性とSIMD化のハイブリッドを
実現することができると考える.\\
よって,仮に変数のグループがn個できた場合でも2n回の試行を行うことで最適な組み合わせを見つけることができる.
( TODO: 論文を引用)


\subsection{実行マシンに依存するパラメータ}
近年のCPUはシングルコアではなく,マルチコアによって計算を並列化することで全体としての計算能力を向上させている.\\
この並列化を行う上で, 本研究では主にOpenMPとMPIを用いたハイブリッド並列に取り組んだ. またハイブリッド並列を行う上で,
OpenMPのスレッド数, MPIのプロセス数そして各スレッドに割り振られるバッファサイズをパラメータとして用いた.\\

% ハイブリッド並列について
\subsubsection{ハイブリッド並列化}
\label{subsubsec:hybrid}
\paragraph{OpenMP}~\\
 NEURONでは, POSIX Threadによるスレッド並列が実装されているが, 京上ではOpenMPによるスレッド並列化しかサポートされていない.
宮本らによる先行研究\cite{miyamoto-master}において, OpenMPベースのバイブリッド並列化機構\cite{hybrid}がNEURONに実装されていたため
本研究ではこのOpenMPが組み込まれているNEURONを用いてシミュレーションを行った.\\
上記の実装ではOMP\_NUM\_THREADSという環境変数の値に応じてスレッドの生成数を変えるものであったため, ジョブスクリプトの内部で\\
\begin{table}[htb]
\begin{center}
\title {OpenMPスレッド数の指定}
{\footnotesize
\begin{framed}
\begin{verbatim}
export OMP_NUM_THREADS=16
\end{verbatim}
\end{framed}
}
\end{center}
\end{table}~\\
とすることでOpenMPのスレッド数を指定することができる.

\paragraph{MPI}~\\
 MPIはほとんどの並列計算機に入っているものであり, NEURONでも特別な変更を加えることなく利用することができる.
京とクラスタで指定する方法は違うものの, 双方ともにジョブスクリプトに利用したいプロセス数を記述することでMPIを使用することができる.\\
クラスタにおいては,
\begin{table}[htb]
\begin{center}
  \title {クラスタ MPIプロセス数の指定}
{\footnotesize
\begin{framed}
\begin{verbatim}
#PBS -l ppn=8
\end{verbatim}
\end{framed}
}
\end{center}
\end{table}~\\
京においては,\\
\begin{table}[htb]
  \begin{center}
  \title {京 MPIプロセス数の指定}
{\footnotesize
\begin{framed}
\begin{verbatim}
#-- proc として指定する値はノード数×プロセス数 --#
#PJM --mpi "proc=8"
\end{verbatim}
\end{framed}
}
\end{center}
\end{table}~\\
\
とすることでMPIのプロセス数を指定することができる.\\

\paragraph{ハイブリッド並列化}~\\
 並列化をする際にOpenMPとMPIと組み合わせることをOpenMPとMPIのハイブリッド並列化という.\\
OpenMPとMPIはそれぞれ長所と短所を持つが, 規模が小さい場合スレッド生成のコストが大きくMPIのみを利用するFlat MPIの性能が
ハイブリッド化するよりも優れていることが多い.\\
 しかしながらノード数が増えていくに連れて, MPIプロセス間での通信に利用されるネットワーク通信部分がボトルネックとなっていく可能性が高くなっていく.
これは単一ノードの持つネットワーク通信に使えるリソースに対し通信対象となるMPIプロセスが増えすぎることが原因となる. そのため計算規模が大きくなるほど, 同じノードの内部ではOpenMPを用いてCPUコア間で共有されているメモリを用いて通信を行い,
外部ノードとの通信にMPIを使うことでボトルネックとなる通信を分散させることができるハイブリッドの強みを生かすことができる.\\
 また京のようにスレッドバリアと呼ばれるスレッドを高速に生成する機構を持っている計算機も存在するため,
OpenMPとMPIを用いる比率を実行マシンに依存するパラメータとして本研究では用い計算機にあった最適化を目指す.\\


\subsubsection{配列サイズの変形}
配列サイズすなわちバッファとして利用するメモリの量も実行マシンに依存するパラメータである.\\
NEURONでの計算では, SIMD化や配列のくくり出しという観点で対象とする変数をベクトル化する場合,
各々の変数に対して最低でもコンパートメント数 × スレッド数の大きさのバッファが必要となる.\\
配列をスタック領域に確保しているため, ヒープ領域に確保するように確保するためにシステムコールを呼ぶ必要はない.
一方で, 計算機上のスタック領域の大きさは限られておりこの領域の大半をバッファとして用いてしまうと実際の計算に支障が出てしまう.\\
特にスレッド数に比例して必要となるバッファが大きくなるため, ハイブリッド並列を行う際に計算機特性特にメモリに関連した性能との関連が大きくなることが予想される.\\
以上から適切な配列サイズを対象となるシミュレーションに応じて選択することで最適化のためのパラメータとして利用することとした.\\


\subsection{コンパイルに関わるパラメータ}
コンパイルオプションについては, 本研究で利用した京とクラスタにおいても大きく異なったため本研究においては
モデルと実行マシンに関連するパラメータを用いた最適化に注力し, コンパイルに利用するコンパイラを変えて比較するにとどまった.\\
デフォルトで利用されているGCCとコンパイル時の最適化に優れているICC( TODO: iccのレファレンス)を比較した.

