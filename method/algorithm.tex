\label{sec:algorithm}
本研究では,モデルに依存するパラメータと実行マシンに依存するパラメータ,そしてプログラムのコンパイル時に関わるパラメータ(コンパイルオプション)を調節することでシミュレーション系の最適化を目指した.\\
以下にそれぞれのパラメータの詳細を示す.\\
\subsection{モデルに依存するパラメータ}
以下にHodgkin-Huxley方程式のモデルを例としてそれぞれのパラメータを示す.\\
モデルに依存するパラメータに関しては先行研究\cite{miyamoto-master}においてSIMD化, 配列構造の最適化により計算速度が大きく向上することが示されているため,
その二つに加え配列構造の順序を入れ替えることによってキャッシュヒット率の向上に取り組んだ.\\
Hodgkin-Huxley方程式は,NEURON内においてMOD形式で次のように記述されている.\\
{\footnotesize
\lstinputlisting[caption=hh.mod,label=mod-hh,frame=single]{src/mod/hh.mod}
}
先行研究の中でも示されている通り,この計算式の中で多くの計算時間を必要とするのはDERIVATIVEであり,
以下のパラメータの多くはDERIVATIVEの計算を行う上でキャッシュヒット率をあげることを目的としている.\\
・宮本さんの論文を参照
また,SIMD化を行う方法としてavxやinline asmなどを利用する方法も存在する.\\
これは現在の課題である汎用性を達成した上で,汎用性を崩さないよう取り組む内容であるため,
今後の課題としたい.\\


( TODO: 例)配列を複数定義し,一つの計算の中で呼び出す場合,Array of Structureでなく
Structure of Arrayとして定義した方が計算が高速化される場合がある.\\
これは以下に示す図によるものである\\
( TODO: 図)
一方で,一つの構造体に多くの変数を定義してしまうとキャッシュラインに入らず,
結果としてキャッシュミスを多発し逆に遅くなるといった問題が生じる.\\
そこで,MODファイル内で定義された計算式の中から変数を抜き出し,
それぞれの変数をUnion-Find木 ( TODO : reference)を用いてグループにまとめ構造体として利用する変数の候補とし,
それぞれの構造体を使うか否かでの高速化を図る.\\
Union-Find木を用いているため,キャッシュという観点ではグループ間の関連はないと言えるため,
これらのグループは独立にシミュレーションを行うことができる.\\
よって,仮に変数のグループがn個できた場合でも2n回の試行を行うことで最適な組み合わせを見つけることができる.
( TODO: SOAの論文を引用)


\subsection{実行マシンに依存するパラメータ}
近年のCPUはシングルコアではなく,マルチコアによって計算を並列化することで全体としての計算能力を向上させている.\\
この並列化を行う上で, 本研究では主にOpenMPとMPIを用いたハイブリッド並列に取り組んだ. またハイブリッド並列を行う上で,
OpenMPのスレッド数, MPIのプロセス数そして各スレッドに割り振られるバッファサイズをパラメータとして用いた.\\

% ハイブリッド並列について
\subsubsection{ハイブリッド並列化}
\paragraph{OpenMP}
NEURONでは, POSIX Threadによるスレッド並列が実装されているが, 京上ではOpenMPによるスレッド並列化しかサポートされていない.
宮本らによる先行研究\cite{miyamoto-master}において, OpenMPベースのバイブリッド並列化機構がNEURONに実装されていたため
本研究ではこのOpenMPが組み込まれているNEURONを用いてシミュレーションを行った.\\
上記の実装ではOMP\_NUM\_THREADSという環境変数の値に応じてスレッドの生成数を変えるものであったため, ジョブスクリプトの内部で
{\footnotesize
\begin{lstlisting}[caption=OpenMPスレッド数の指定,label=cluster-job-example,numbers=none]
export OMP_NUM_THREADS=16
\end{lstlisting}
}
とすることでOpenMPのスレッド数を指定することができる.\\

\paragraph{MPI}
MPIはほとんどの並列計算機に入っているものであり, NEURONでも特別な変更を加えることなく利用することができる.\\
京とクラスタで指定する方法は違うものの, 双方ともにジョブスクリプトに利用したいプロセス数を記述することでMPIを使用することができる.\\
クラスタにおいては,
{\footnotesize
\begin{lstlisting}[caption=クラスタ MPIプロセス数の指定,label=cluster-mpi-num-process,numbers=none]
#PBS -l ppn=8
\end{lstlisting}
}
京においては,
{\footnotesize
\begin{lstlisting}[caption=京 MPIプロセス数の指定,label=k-mpi-num-process,numbers=none]
#PJM --mpi "proc=8"
\end{lstlisting}
}
とすることでMPIのプロセス数を指定することができる.\\

\paragraph{ハイブリッド並列化}
並列化をする際にOpenMPとMPIと組み合わせることをOpenMPとMPIのハイブリッド並列化という.\\
OpenMPとMPIはそれぞれ長所と短所を持つが, 規模が小さい場合スレッド生成のコストが大きくMPIのみを利用するFlat MPIの性能が
ハイブリッド化するよりも優れていることが多い.\\
しかしながらノード数が増えていくに連れて, MPIプロセス間での通信に利用されるネットワーク通信部分がボトルネックとなっていく可能性が高くなっていく.
これは単一ノードの持つネットワーク通信に使えるリソースに対し通信対象となるMPIプロセスが増えすぎることが原因となる.\\
そのため計算規模が大きくなるほど, 同じノードの内部ではOpenMPを用いてCPUコア間で共有されているメモリを用いて通信を行い,
外部ノードとの通信にMPIを使うことでボトルネックとなる通信を分散させることができるハイブリッドの強みを生かすことができる.\\

( TODO: 図)

また京のようにスレッドバリアと呼ばれるスレッドを高速に生成する機構を持っている計算機も存在するため,
OpenMPとMPIを用いる比率を実行マシンに依存するパラメータとして本研究では用い計算機にあった最適化を目指す.\\


\subsubsection{配列サイズの変形}


\subsection{コンパイルに関わるパラメータ}
コンパイルオプションについては, 本研究で利用した京とクラスタにおいても大きく異なったため本研究においては
モデルと実行マシンに関連するパラメータを用いた最適化に注力し, コンパイルに利用するコンパイラを変えて比較するにとどまった.\\
 またコンパイラの比較は\ref{subsec:compare}節において,
自動最適化を行ったものに対してコンパイル時の最適化に優れているIntel C++ Compiler(以下ICC)\cite{icc}を利用してコンパイルしたNEURON
をベンチマークとして用いることで行った.

