本研究では,モデルに依存するパラメータと実行マシンに依存するパラメータ,そしてプログラムのコンパイル時に関わるパラメータ(コンパイルオプション)を調節することでシミュレーション系の最適化を目指した.\\
以下にそれぞれのパラメータの詳細を示す.\\
\subsection{モデルに依存するパラメータ}
以下にHodgkin-Huxley方程式のモデルを例としてそれぞれのパラメータを示す.\\
モデルに依存するパラメータに関しては先行研究 ( TODO: add reference)においてSIMD化, 配列構造の最適化により計算速度が大きく向上することが示されているため,
その二つに加え配列構造の順序を入れ替えることによってキャッシュヒット率の向上に取り組んだ.\\
Hodgkin-Huxley方程式は,NEURON内においてMOD形式で次のように記述されている.\\
{\footnotesize
\lstinputlisting[caption=hh.mod,label=mod-hh,frame=single]{src/mod/hh.mod}
}
先行研究の中でも示されている通り,この中でプロファイル結果から多くの計算時間を必要とするのはDERIVATIVE ( TODO: reference)
であり以下のパラメータの多くはDERIVATIVEの計算を行う上でキャッシュヒット率をあげることを目的としている.\\
・宮本さんの論文を参照
また,SIMD化を行う方法としてavxやinline asmなどを利用する方法も存在する.\\
これは現在の課題である汎用性を達成した上で,汎用性を崩さないよう取り組む内容であるため,
今後の課題としたい.\\

・変数の配列化によるメモリアクセスの連続化
( TODO: 例)配列を複数定義し,一つの計算の中で呼び出す場合,Array of Structureでなく
Structure of Arrayとして定義した方が計算が高速化される場合がある.\\
これは以下に示す図によるものである\\
( TODO: 図)
一方で,一つの構造体に多くの変数を定義してしまうとキャッシュラインに入らず,
結果としてキャッシュミスを多発し逆に遅くなるといった問題が生じる.\\
そこで,MODファイル内で定義された計算式の中から変数を抜き出し,
それぞれの変数をUnion-Find木 ( TODO : reference)を用いてグループにまとめ構造体として利用する変数の候補とし,
それぞれの構造体を使うか否かでの高速化を図る.\\
Union-Find木を用いているため,キャッシュという観点ではグループ間の関連はないと言えるため,
これらのグループは独立にシミュレーションを行うことができる.\\
よって,仮に変数のグループがn個できた場合でも2n回の試行を行うことで最適な組み合わせを見つけることができる.
( TODO: SOAの論文を引用)

・配列の構造変形(時間があれば)なければここを消す
\subsection{実行マシンに依存するパラメータ}
近年のCPUはシングルコアではなく,マルチコアによって計算を並列化することで全体としての計算能力を向上させている.\\
一方で,この並列化を行う上でのパラメータは実行するマシンごとに依存するものである.\\
ここで主に対象としたパラメータはOpenMPのスレッド数とMPIのプロセス数である.\\
\subsubsection{スレッド数}
OpenMPのスレッドに関与するパラメータに関する説明 ( TODO : わあああ)
\subsubsection{プロセス数}
MPIプロセスに関与するパラメータに関する説明 ( TODO : わあああい)
\subsubsection{配列サイズの変形}
\subsubsection{配列サイズの変形}

\subsection{コンパイルに関わるパラメータ}
TODO: 時間があれば記述する
