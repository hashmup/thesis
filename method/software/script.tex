\subsection{環境設定スクリプト}
本研究で作成したソフトウェアは, 内部で使用しているPythonのライブラリがPython3以降を必要としているため,
実行するPythonのバージョンの制約を受ける.\\
そのため京に代表されるPython3が存在せず, 管理者権限を持つことができない環境においては,
管理者権限を必要とせずユーザーの権限の中でPythonのプログラムを実行できる環境を構築する必要がある.\\
この問題を解決するため, Pyenv\cite{pyenv-repo}というPython専用の仮想環境を構築するライブラリを利用し,
仮想環境上でソフトウェアを実行することにした.\\
この方法を用いることで, 実行する計算機によってソフトウェアを変更する必要はなくなったが,
実行するための環境を構築する手間がかかるため,以下に示す環境構築スクリプトを作成した.\\
またMakefileを作成し, makeコマンドから環境構築のスクリプトの呼び出しやシミュレーションの実行を行うようにした.\\
\begin{table}[htb]
  \caption {環境設定に利用するmakeコマンド}
{\footnotesize
\begin{framed}
\begin{verbatim}
# 環境構築を行うコマンド
# scripts/setup_env_and_install_libraries.sh を実行する
make install

# scripts/pull_required_projects.sh を実行する
make pull

# make install, make pull の順で実行する
make setup

# シミュレーションを行うコマンド
# Makefile と同じディレクトリにある job.json を読み込みシミュレーションを実行する
make run

# ファイル名で指定した MOD ファイルを C 言語のファイルに変換する
make compile "ファイル名"
\end{verbatim}
\end{framed}
}
\end{table}
\\
\begin{table}[htb]
  \caption {環境構築スクリプトのフォルダ構成}
{\footnotesize
\begin{framed}
\begin{verbatim}
  Makefile
  scripts/
     |--- setup_env_and_install_libraries.sh
     |--- pull_required_projects.sh
\end{verbatim}
\end{framed}
}
\end{table}
\\
\begin{table}[htb]
  \caption {必要なライブラリ}
{\footnotesize
\begin{framed}
\begin{verbatim}
Python
    pandas: シミュレータでパラメータやそれぞれのパラメータに対応する結果を保持するデータ管理ライブラリ
    textX: トランスパイラで MOD ファイルから抽象木を生成する際に利用するメタ言語を作成するためのライブラリ
    jinja2: トランスパイラで C 言語のファイルを生成する際に利用するテンプレートライブラリ

NEURON
    neuron_kplus: 宮本らの先行研究によって追加実装された NEURON 本体と, そのインストールスクリプト等を含んだソース群
\end{verbatim}
\end{framed}
}
\end{table}
\\

{\footnotesize
\lstinputlisting[caption=setup\_env\_and\_install\_libraries.sh,label=setup_env,frame=single]{src/script/setup_env_and_install_libraries.sh}
}

{\footnotesize
\lstinputlisting[caption=pull\_required\_projects.sh,label=pull_project,frame=single]{src/script/pull_required_projects.sh}
}
