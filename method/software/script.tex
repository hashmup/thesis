作成したシミュレータ・トランスパイラはPython( TODO: reference)のモジュールとして作成したが,
pip( TODO: reference )のようなモジュール管理ツールが存在しない環境(スーパーコンピュータ京)においては,
モジュールとして公開するだけでは不十分である.\\
特にスーパーコンピュータ京では,デフォルトのPython( TODO: reference)のバージョンが2.6.6,
sudo権限を有しないため外部プログラムのインストールが難しいという環境であったため,
Pyenvを利用して汎用的な環境を作成することにした.\\
以下作成したスクリプトの概要とその用途を示す.\\
・Makefile\\
このプロジェクトのMakefile( TODO: reference).\\
主に利用するのは以下の3つのコマンド\\
{\footnotesize
\begin{lstlisting}[numbers=none]
# scripts/setup_env_and_install_libraries.shを実行する
make install

# scripts/pull_required_projects.shを実行する
make pull

# make install, make pullの順で実行する
make setup
\end{lstlisting}
}
・scripts/\\
Makefileで実際に実行されるシェルスクリプト群であり,NEURON本体のインストールと本研究で用いるPythonの環境を整える役割を担っている.\\
以下に示したように,Pyenvを用いてPython3 ( TODO: reference)のインストールと実行の際に必要となるライブラリをインストールしている.\\
( TODO: ライブラリの説明)
\indent {\footnotesize
\lstinputlisting[caption=setup\_env\_and\_install\_libraries.sh,label=setup_env,frame=single]{src/script/setup_env_and_install_libraries.sh}
}
次に示すように,本研究の自動チューニングの対象であるNEURONのインストールを行っている.\\
NEURONのディレクトリが存在しない場合は,Githubからクローンをした後に必要なディレクトリの追加を行う.\\
一方で,存在する場合は最新のものへの更新を行っている.\\
\indent ・pull\_required\_projects.sh\\
\indent {\footnotesize
\lstinputlisting[caption=pull\_required\_projects.sh,label=pull_project,frame=single]{src/script/pull_required_projects.sh}
}
