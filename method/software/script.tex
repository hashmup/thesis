作成したシミュレータ・トランスパイラはPython( TODO: reference)のモジュールとして作成したが,
pip( TODO: reference )のようなモジュール管理ツールが存在しない環境(スーパーコンピュータ京)においては,
モジュールとして公開するだけでは不十分である.\\
特にスーパーコンピュータ京では,デフォルトのPython( TODO: reference)のバージョンが2.6.6,
sudo権限を有しないため外部プログラムのインストールが難しいという環境であったため,
Pyenvを利用して汎用的な環境を作成することにした.\\
以下作成したスクリプトの概要とその用途を示す.\\
・Makefile\\
このプロジェクトのMakefile( TODO: reference).\\
主に利用するのは以下の3つのコマンド\\
\indent ・make install\\
\indent \indent このコマンドでは以下に示すscripts/setup\_env\_and\_install\_libraries.shを実行する.\\
\indent ・make pull\\
\indent \indent このコマンドでは以下に示すscripts/pull\_required\_projects.shを実行する.\\
\indent ・make setup\\
\indent \indent 上記の二つのコマンドをinstall,pullの順で実行する.\\
・scripts/\\
\indent ・setup\_env\_and\_install\_libraries.sh\\
\indent \indent このスクリプトは大きくわけ2つのことをする.\\
\indent \indent ・Pyenvのインストール\\
\indent \indent ・genieで必要となるPythonのライブラリをインストール\\
\indent ・pull\_required\_projects.sh\\
\indent \indent このスクリプトはシミュレーションソフトであるNEURON環境を構築することを目的としている.\\
