先行研究( TODO: ref)では,モデルに依存するパラメータを調節するために,
計算モデルが記述されたMODファイルからnmodlを介して生成されたCファイルを手動で変更を加えることで最適化を図っていた.\\
本研究では,自動チューニングを目的としているため,このプロセスも自動化する必要があり,そのためにこのMODからCへ変換するトランスパイラを作成した.\\
MODをパースするにあたってはDomain-Specific Languagesを作成するためのPythonライブラリである,textX ( TODO: ref)を利用した.\\
また,MODのContext Free GrammarはMODファイルからNeuroMLを生成するためのプロジェクトであるpynmodl ( TODO: ref)のプログラムを用いた.\\

\subsubsection{nmodl}
トランスパイラを作成するにあたり参考にしたNEURONに付属しているトランスパイラであるnmodlについて述べる.\\
コードがひどかったです...

\subsubsection{アルゴリズム}

\subsubsection{実装}
