最適化の方法として,複数のパラメータからモデル,実行環境に即したパラメータを選択するという手法を選択したが,そのためには複数のパラメータでシミュレーションを行いその結果を集約する
プログラムが必要となる.\\
本研究ではこのパラメータ選択を容易かつ高速に行うため以下に示すプログラムを作成した.\\
・MODファイルからパラメータとなりうる変数を自動で抽出し,それぞれの関係性を元に配列とその順序の候補を生成する.\\
・ジョブキューのシステムを持っているマシンにおいて,複数のジョブを並行して投げ結果を非同期的に集約できる.\\
・実行結果を最適化前のデフォルトの結果と比較し,実行結果に対して影響がないかを確認する.\\
・json形式で実行するファイル,各パラメータの範囲(プロセス数は1から10など)を指定することができる.\\
\subsubsection{全体構成}
パラメータ探索の詳細を述べる前に作成したシミュレータプログラムの概略を疑似コードにて示す.\\
{\footnotesize
\lstinputlisting[caption=シミュレータ疑似コード,label=hogehoge,frame=single]{src/pseudocode/simulator}
}
この疑似コードは大きく3つの機能から成立しているため,次にその詳細を示す.\\
\paragraph{全体ループ}
疑似コード内の8-22行目までが全体のループを構成している.\\
{\footnotesize
\lstinputlisting[caption=シミュレータ全体ループ,label=hogehoge,frame=single, firstline=8, lastline=22]{src/pseudocode/simulator}
}
\paragraph{ジョブ実行}
\paragraph{ジョブ結果の集約}

\subsubsection{モデルに依存するパラメータ}
