\subsubsection{配列構造}
\label{subsubsec:soa}
配列を複数定義し一つの計算式の中でそれらを同時に呼び出す場合, 空間的局所性が低くなりキャッシュミスを多く生じさせる可能性がある.
そのため同時に利用する配列達を一つの配列としてくくりだすことで, 空間的局所性を高くし高速化を図る手法が考えられる.\\
\begin{table}[htb]
{\footnotesize
\begin{framed}
\begin{verbatim}
int a[100], b[100], c[100], d[100];
for i=0 to 100 {
  d[i] = a[i] + b[i] + c[i];
}
\end{verbatim}
\end{framed}
}
\end{table}
\\
というプログラムを例とすると,\\
\begin{table}[htb]
{\footnotesize
\begin{framed}
\begin{verbatim}
int abcd[100][4];
for i=0 to 100 {
  abcd[i][3] = abcd[i][0] + abcd[i][1] + abcd[i][2];
}
\end{verbatim}
\end{framed}
}
\end{table}
\\
とすることで図\ref{fig:merge-array}に示すように連続した領域にアクセスさせることができるようになり, キャッシュ効率の向上が見込まれる.\\
一方でSIMD化の観点ではa, b, c, dがそれぞれ独立した形の配列でなくなってしまい,
SIMD命令を使いにくくなる可能性が非常に高い.\\

\begin{figure}[htb]
% h:here, t:top, b:bottom, p:page
  \begin{center}
    \includegraphics[width=10cm]{./images/MERGE-ARRAY.pdf}
    \caption{配列のくくりだし}
    \label{fig:merge-array}
  \end{center}
\end{figure}~\\
以上から空間局所性とSIMD化という最適化を行う上で大変重要な要素どちらか一方だけを考えるのではなく,
適切なハイブリッド構造を用いることを目指した.\\
 まず演算に利用する変数の数が多い時はSIMD化は難しいと考えられる. これはSIMD演算器のビット数に依存する問題だが倍精度の演算をする場合
1-4変数(double型64bitsに対しSIMD演算器は64-256bitsが一般的)の演算を並列処理できるが, 変数の数がより多い場合は
同時に実行することが不可能であるからである.\\
 この場合, 配列をくくり出すことによって空間局所性を高くする方がより計算処理の高速化を実現できると考えられる.\\

\paragraph{Union-Find木の利用}~\\
 配列をくくり出す際, すべての配列をまとめてしまうとSIMD化できる部分を見逃してしまったり,
要素数が多すぎるためキャッシュラインに入りきらなくなるといった問題が発生する.
そのためくくり出す配列は計算上関連のあるもの同士にするべきであり, 関連のある変数をグループ化するためにUnion-Find木を利用した.\\
 MODファイルに定義された一つの式の変数を相互に関連する変数であると見なしUnion-Find木に追加していくことで, 仮に次のような計算式がMODファイルに存在していた場合
\begin{eqnarray}
  a &=& b * c\\
  d &=& e * f\\
  g &=& a + h
\label{eq:3x0}
\end{eqnarray}
(a, b, c, g, h)と(d, e, f)という変数のグループを作成することができる.\\

\paragraph{配列のくくり出し実装}~\\

以上を踏まえ,SIMD化と配列のくくりだしのハイブリッドを実現するために以下のアルゴリズムを実装した.\\

{\footnotesize
\lstinputlisting[caption=SIMD化と配列のくくりだしのアルゴリズム 疑似コード,label=pseudocode-simd-soa,frame=single]{src/pseudocode/simd-soa}
}~\\
 こうしてUnion-Find木で作成されたグループは, 空間局所性またはSIMD化のどちらかがより有効に働くことから,
それぞれのグループを配列としてくくり出すか否かを独立に試行することで空間局所性とSIMD化のハイブリッドを
実現することができると考える.\\
