( TODO: 例)配列を複数定義し,一つの計算の中で呼び出す場合,Array of Structureでなく
Structure of Arrayとして定義した方が計算が高速化される場合がある.\\
これは以下に示す図によるものである\\
( TODO: 図)
一方で,一つの構造体に多くの変数を定義してしまうとキャッシュラインに入らず,
結果としてキャッシュミスを多発し逆に遅くなるといった問題が生じる.\\
そこで,MODファイル内で定義された計算式の中から変数を抜き出し,
それぞれの変数をUnion-Find木 ( TODO : reference)を用いてグループにまとめ構造体として利用する変数の候補とし,
それぞれの構造体を使うか否かでの高速化を図る.\\
Union-Find木を用いているため,キャッシュという観点ではグループ間の関連はないと言えるため,
これらのグループは独立にシミュレーションを行うことができる.\\
よって,仮に変数のグループがn個できた場合でも2n回の試行を行うことで最適な組み合わせを見つけることができる.
( TODO: SOAの論文を引用)
