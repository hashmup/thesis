前節の配列構造において,複数の配列を一つの構造体としてまとめた際に,
メモリサイズが大きすぎる場合キャッシュラインに乗り切らず逆にキャッシュミスが多発するという問題について述べた\\
そのため,前節のアルゴリズムによって有効だとされた構造体配列の中でさらにそれぞれの変数を利用するか否か,
さらに配列のために確保するメモリサイズを変更することでキャッシュミスが減少するかをためす.\\
マシンに応じてメモリのサイズが異なるため,キャッシュ利用効率はこの配列のために確保するメモリサイズに大きく依存すると考えられる.\\
