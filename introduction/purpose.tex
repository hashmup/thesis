\subsection{研究の目的と手法}
高速化・最適化への需要への項で述べたように,本研究は個々の神経細胞のイオンチャンネルモデルに対し汎用的な最適化手法を開発することである.\\
神経回路シミュレーションを行うソフトウェアは多数存在するが,本研究では先行研究で用いられていたNEURONというソフトウェアを利用する.\\
NEURONでは,神経細胞のモデルとそれぞれの細胞の関係を微分方程式の形でモデルファイル(MODファイル)として記述することができ,
nmodlというトランスパイラがMODファイルをCファイルに変換することで実行している.\\
先行研究では,この生成されたCファイルに着目し手動での最適化を行っていたが,本研究ではCファイルの生成と実行,結果の集約を自動で行うことで
複数のパラメータを試し,シミュレーションを実行する上で最適なパラメータを選択することを目指す.\\
