\subsection{神経科学においてシミュレーションを行う意義}
脳機能の理解を目的として,スーパコンピュータを用いた神経回路のシミュレーションが行われている.
また,消費電力やシミュレーションの割り当て時間といったリソースの問題やリアルタイムデータ同化への
需要からシミュレーションの高速化・最適化が求められている.
しかし,神経細胞には様々な種類のものが存在するため,
個々の神経細胞のイオンチャンネルのモデルを最適化された形で実装するために,
これまでそれぞれのモデルに対して多大な努力が行われてきた.
また,現代の計算機にも多様な種類が存在し,それぞれに対する最適化も個別に行われてきた.
本研究の目的はそれぞれの細胞モデルのシミュレーションコードを個々のアーキテクチャに合わせて,
自動又は半自動的に最適化を行う手法を確立することである.

\subsection{神経回路シミュレーションの高速化・最適化への需要}
