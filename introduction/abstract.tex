\subsection{神経科学においてシミュレーションを行う意義}
神経科学分野において生物の脳機能を解明することは主たる目的である.
それは, 人間を含む生物の知能の理解という人間の本源的な要求の発露のみならず,
情報処理プログラムのアルゴリズムやロボット, マンマシンインターフェースのような工学応用,
アルツハイマー病やパーキンソン病といった疾患の治療や脳の心の健康科学, さらにニューロンエンハンスメントによる知能増強まで見通す医学分野の発展にも大きな貢献をすることが期待される.
一方で, 生物の脳では非常に短い時間で膨大な情報が処理されており例えば人間の脳の情報処理は$10^{21}FLOPS$だという見積もりがある\cite{science-roadmap}.
これは非常に大きな値にみえるが, コンピュータの能力は年々進歩をつづけており,
日本においても京コンピュータにおいて$10^{16}FLOPS$が2012年に達成されて,
2020年に$10^{18}FLOPS$を目指すポスト京の開発が進んでいることを考えればスーパーコンピュータの利用が前提となるとはいえ生物の脳全体のシミュレーションも神経科学の視野にはいっているといえる.
トップダウン的な脳の構築原理が明らかになってない以上,様々な生物実験の結果を微視的なレベルから入れ込んでいく ボトムアップアプローチによってシミュレーションを行い,
しかるのちに現実には観察が難しい脳機能の細部までもシミュレーションを介してでも観察できる環境を構築することは大きな意義を持つ.



\subsection{神経回路シミュレーションの高速化・最適化への需要}
しかしながら生物の脳機能は元来単純な一つモデルで表すことができるものではなく,
その機能の解明には多数のモデルを混在させることが必要になるであろうと考えられる.
現在modelDB\cite{model-db}をはじめとして, 脳機能・イオンチャネルのモデルは多くプログラムとして蓄積されているが,
それらは実行速度の観点からチューニングがなされていないものであり,スーパーコンピュータのような高価な計算資源でシミュレーションを行おうとする場合, 限られた資源を有効に活用するため高速化することが望ましい.
一方で, 多数のモデルを一つのシミュレーションが含むことを想定すれば,
それらのモデルすべてに対し手動で高速化を行うには膨大な時間と労力を必要とする.
もしくは対象のモデルが一つであってもモデリングする人間が計算チューニングに慣れているとは限らず,
そのため多くのコストがかかる可能性がある.
それゆえ, 実験データから脳機能を再構築するボトムアップアプローチを取るにあたり,
モデルを自動的に高速化する手段を開発することは大きな意義を持つ.
そこで,本研究では個々のイオンチャンネルモデルを, シミュレーションを実行する計算機に合わせて自動で最適化するソフトウェアを作成することで,
これまで人の手で逐次行われてきた最適化の汎用化を目指す.
