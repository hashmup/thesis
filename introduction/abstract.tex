\subsection{神経科学においてシミュレーションを行う意義}
・ボトムアップアプローチ\\

・また,当研究室ではそうしたモデル構築のために実験などを行い,そうしたデータを元に様々なシミュレーション系を構築してきた\\
脳機能の理解を目的として,スーパコンピュータを用いた神経回路のシミュレーションが行われている.
また,消費電力やシミュレーションの割り当て時間といったリソースの問題やリアルタイムデータ同化への
需要からシミュレーションの高速化・最適化が求められている.

また,現代の計算機にも多様な種類が存在し,それぞれに対する最適化も個別に行われてきた.
本研究の目的はそれぞれの細胞モデルのシミュレーションコードを個々のアーキテクチャに合わせて,
自動又は半自動的に最適化を行う手法を確立することである.

\subsection{神経回路シミュレーションの高速化・最適化への需要}
・神経回路シミュレーションには非常に大きな計算力が必要である
一方で,こうした神経回路シミュレーションには非常に大きな計算能力が必要とされてきた.\\
本研究はスーパーコンピュータ京に関連するポスト京プロジェクトの一環として行われているが,\\
スーパーコンピュータを用いてもなお計算には多くの時間がかかっている.\\
そうした状態を踏まえ,系の構築だけでなくシミュレーション自体の高速化・最適化が求められている.\\
・神経回路シミュレーションの最適化の難しさ
しかし,神経細胞には様々な種類のものが存在するため,
個々の神経細胞のイオンチャンネルのモデルを最適化された形で実装するために,
これまでそれぞれのモデルに対して多大な努力が行われてきた.
・本研究の意義
そこで,本研究では個々のイオンチャンネルモデルを自動で最適化するソフトウェアを作成することで,
これまで人の手で逐次行われてきた最適化の汎用化を目指す.
